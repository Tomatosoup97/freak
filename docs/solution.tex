\documentclass{article}
\usepackage[utf8]{inputenc}
% \usepackage{polski}
\usepackage[]{amsthm}
\usepackage[]{amssymb} %gives us the character \varnothing

\usepackage{hyperref}
\usepackage{titling}
\usepackage[backend=bibtex]{biblatex}
\addbibresource{mybib.bib}
\theoremstyle{definition}
\newtheorem{definition}{Definition}

\theoremstyle{lemma}
\newtheorem{lemma}{Lemma}

\theoremstyle{observation}
\newtheorem{observation}{Observation}

\theoremstyle{theorem}
\newtheorem{theorem}{Theorem}

\newcommand*{\fullref}[1]{\hyperref[{#1}]{\autoref*{#1} \nameref*{#1}}}


\title{Continuation Passing Style for Effect Handlers}
\author{Mateusz Urbańczyk}
\date{7 January 2020}

\begin{document}
\maketitle

\begin{abstract}

    We provide an implementation of algebraic effects and handlers by doing continuation
    passing style tranformation of the functional programming language Freak,
    which is based on the existing Links langauge \cite{handlers-cps}.

\end{abstract}

\section{Introduction}

Implementation of Continuation Passing Style for Effect Handlers paper \cite{handlers-cps}

\section{State of the art}

    Frank \cite{frank}, Koka \cite{leijen-koka}, Helium \cite{helium}

\section{Syntax}

\section{Operational semantics}

    \subsection{Integers}

    \subsection{Binary operators}

\section{Usage guide}
    % TODO: not sure if correct place for that

    All commands are available within \verb src  directory.

    \subsection{Build and install}

    \begin{itemize}
        \item Install dependencies: \verb!make install!
        \item Compile: \verb!make build!
        \item Link to PATH: \verb!sudo make link!
        \item Remove artiacts: \verb!make clean!
        \item Run tests: \verb!make tests!
        \item Run code linter: \verb!make lint!
    \end{itemize}

    After compiling and linking program to PATH, one may evaluate program as
    follows: \verb!freak programs/choicesList.fk!. The actual code is described in \fullref{sec:choice-example}

    \subsection{Running tests}

    Test cases are available \href{https://github.com/Tomatosoup97/freak/blob/master/src/Tests.hs}{\underline{here}},
    they include both inline and file-based tests. For more details about
    writing tests, one may refer to HUnit documentation \cite{hunit-docs}. \\

    \begin{itemize}
        \item Run tests: \verb!make tests!
        \item Run code linter: \verb!make lint!
        \item Compile, run linter and tests: \verb!make check!
    \end{itemize}

\section{Implementation}

    The Freak implementation is available at \href{https://github.com/Tomatosoup97/freak}{https://github.com/Tomatosoup97/freak}.
    CPS based on \cite{appel-continuations}

    \subsection{Row types}

    \subsection{Type inference}

    Type system as of this day is not implemented, as the focus has been put
    on CPS transformation. Further work is required.

\section{Examples}

    In this section we present a few examples to show the capabilities of the
    language. The ideas has been based on \cite{eff}, and thus will not be
    described in great details.

    \subsection{Choice}
    \label{sec:choice-example}

    The first example will be based on modelling (nondeterministic) choice
    in the program. We will make two decisions, which will affect the computation
    result:

    \begin{verbatim}
    let c1 <- do Choice () in
    let c2 <- do Choice () in
    let x <- if c1 then return 10 else return 20 in
    let y <- if c2 then return 0 else return 5 in
        return x - y
    \end{verbatim}

    With that in hand, we may want to define effect handlers:

    \begin{verbatim}
    handle ... with {
        Choice p r ->
            let t <- r 1 in
            let f <- r 0 in
            <PLACEHOLDER> |
        return x -> return x
    }
    \end{verbatim}

    where in the \verb!<PLACEHOLDER>! we can define on what to do with the
    computation. For example, min-max strategy for picking the minimum value:

    \begin{verbatim}
    if t < f then return t else return f
    \end{verbatim}

    where the code evaluates to \verb!5!. Another example is a handler that
    collects all possible results, which can be achieved by putting
    \verb!return (t, f)! in the \verb!<PLACEHOLDER>!, which evaluates to \verb!((10, 5), (20, 15))!.

    \subsection{Exceptions}

    Exceptions are simply algebraic effect handlers which drop the resumption.

    \begin{verbatim}
    handle
        if x == 0 then do Raise ()
                  else return 1 / x
    with {
        Raise p r -> return 42 |
        return x -> return x
    }
    \end{verbatim}

    Where we imagine that x variable has been bound previously. More exemplary
    programs in Freak language can be found \href{https://github.com/Tomatosoup97/freak/tree/master/src/programs}{\underline{here}}.

\section{Future work}

    \subsection{Abstract machine}

    Implement abstract machine described in \cite{liberating-effects}

    \subsection{Multiple instances of algebraic effect}

    Proposed Helium \cite{binders-labels}, or Eff \cite{eff}

    \subsection{Make the langugae more usable}

    While the language is turing-complete, for convenient usage it
    requires more basic constructs and syntactic sugar for common patterns
    that would ease the programming.

    \subsection{Selective CPS}

    As in \cite{leijen-koka}

    \subsection{Exceptions as separate constructs}

    Exceptions are a trivial example of algebraic effect where the resumption is
    discarded, and as described in \S 4.5 \cite{handlers-cps}, they can be modeled
    as a separate construct to improve performance.

    \subsection{Shallow handlers}

    X \cite{shallow-handlers}

\section{Conclusions}

\printbibliography

\end{document}
