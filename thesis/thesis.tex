% Opcje klasy 'iithesis' opisane sa w komentarzach w pliku klasy. Za ich pomoca
% ustawia sie przede wszystkim jezyk i rodzaj (lic/inz/mgr) pracy, oraz czy na
% drugiej stronie pracy ma byc skladany wzor oswiadczenia o autorskim wykonaniu.
\documentclass[declaration,shortabstract]{iithesis}

\usepackage[utf8]{inputenc}


%%%%% DANE DO STRONY TYTUŁOWEJ
% Niezaleznie od jezyka pracy wybranego w opcjach klasy, tytul i streszczenie
% pracy nalezy podac zarowno w jezyku polskim, jak i angielskim.
% Pamietaj o madrym (zgodnym z logicznym rozbiorem zdania oraz estetyka) recznym
% zlamaniu wierszy w temacie pracy, zwlaszcza tego w jezyku pracy. Uzyj do tego
% polecenia \fmlinebreak.
\polishtitle    {Efekty koalgebraiczne oraz ich kohandlery \fmlinebreak w językach programowania}
\englishtitle   {Coalgebraic effects and their cohandlers \fmlinebreak in programming languages}
\polishabstract {
    \ldots
}
\englishabstract{
    \ldots
}
\author         {Mateusz Urbańczyk}
\advisor        {dr Maciej Piróg}
\date           {30 czerwca 2020}                     % Data zlozenia pracy
% Dane do oswiadczenia o autorskim wykonaniu
% \transcriptnum {291480}                     % Numer indeksu
% \advisorgen    {dr. Macieja Piróga} % Nazwisko promotora w dopelniaczu
%%%%%


%%%%% WLASNE DODATKOWE PAKIETY
% \usepackage{graphicx,listings,amsmath,amssymb,amsthm,amsfonts,tikz}
\usepackage[backend=bibtex]{biblatex}
\addbibresource{mybib.bib}
%
%%%%% WŁASNE DEFINICJE I POLECENIA
%
%\theoremstyle{definition} \newtheorem{definition}{Definition}[chapter]
%\theoremstyle{remark} \newtheorem{remark}[definition]{Observation}
%\theoremstyle{plain} \newtheorem{theorem}[definition]{Theorem}
%\theoremstyle{plain} \newtheorem{lemma}[definition]{Lemma}
%\renewcommand \qedsymbol {\ensuremath{\square}}
% ...
%%%%%

\begin{document}

%%%%% POCZĄTEK ZASADNICZEGO TEKSTU PRACY

\chapter{Introduction}
\section{Problem Analysis}
\section{Problem Statement}
\section{Thesis Outline}

\ldots \cite{frank}

\chapter{Background}
\section{Algebraic Effects}
\section{Categorical Setting for Universal Algebra}
\section{Comodels and Duality}

\chapter{Proposed Solutions}
\section{Dynamic Constraints Checking}
\section{Linear Types}
\section{Data-Flow Analysis}
\section{Cohandlers as a Separate Construct}
\section{Related Work}

\chapter{Coeffectful Programming with Cohandlers}
\section{Examples}
\section{Usage guide}

\chapter{Calculus of Freak language}
\section{Syntax}
\section{Typing Rules}
\section{Operational Semantics}
\section{Continuation Passing Style Transformation}

\chapter{Implementation}
\section{Abstract Syntax Trees}
\section{Curried Translation}
\section{Uncurried Translation}
\section{Cohandlers}
\section{Type System}
\section{Source Code Structure}

\chapter{Conclusion}
\section{Summary}
\section{Future work}

%%%%% BIBLIOGRAFIA

% \begin{thebibliography}{1}
% \bibitem{example} \ldots
% \end{thebibliography}

\printbibliography

\end{document}
