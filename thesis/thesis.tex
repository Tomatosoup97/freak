\documentclass[declaration,shortabstract]{iithesis}

\usepackage[utf8]{inputenc}



%%%%% DANE DO STRONY TYTUŁOWEJ
\polishtitle{Efekty koalgebraiczne oraz ich kohandlery \fmlinebreak{} w językach programowania}
\englishtitle{Coalgebraic effects and their cohandlers \fmlinebreak{} in programming languages}
\polishabstract{
    \ldots
}
\englishabstract{
    \ldots
}
\author{Mateusz Urbańczyk}
\advisor{dr Maciej Piróg}
\date{1 września 2020}                     % Data zlozenia pracy
% Dane do oswiadczenia o autorskim wykonaniu
% \transcriptnum {291480}                     % Numer indeksu
% \advisorgen    {dr. Macieja Piróga} % Nazwisko promotora w dopelniaczu
%%%%%


%%%%% WLASNE DODATKOWE PAKIETY
% \usepackage{graphicx,listings,tikz}
\usepackage{syntax}
\usepackage{mathalfa}
\usepackage{textcomp}
\usepackage{stmaryrd}
\usepackage{tikz-cd}
% Not sure if necessary
\usepackage[utf8]{inputenc}
\usepackage{rotchiffre}



\usepackage[backend=bibtex]{biblatex}
\addbibresource{mybib.bib}
%
%%%%% WŁASNE DEFINICJE I POLECENIA
%
\theoremstyle{definition} \newtheorem{definition}{Definition}[chapter]
\theoremstyle{remark} \newtheorem{remark}[definition]{Observation}
\theoremstyle{plain} \newtheorem{theorem}[definition]{Theorem}
\theoremstyle{plain} \newtheorem{lemma}[definition]{Lemma}
%\renewcommand \qedsymbol {\ensuremath{\square}}

\newcommand{\mathVar}[1]{{\operatorname{\mathit{#1}}}}

% ...
%%%%%

\begin{document}


\chapter{Introduction}

\textit{
    My algebraic methods are really methods of working and thinking; this is why
    they have crept in everywhere anonymously. $\sim{}$Emmy Noether
}

% TODO This requires a rework

In this thesis we present experimental programming language Freak, which is an
implementation of Continuation Passing Style for Effect Handlers paper~\cite{handlers-cps},
with the addition of a few basic constructs. We start by presenting the related work,
then discuss syntax and operational semantics. Basic usage guide for playing with the
language is provided, as well as implementation details and examples. We conclude by
stating what are the possible augmentations, that are intended to be made in the future.

\section{Problem Analysis}

What are the issues with normal algebraic effects?

\section{Problem Statement}

Brief and concise description of the problem

\section{Thesis Outline}

\chapter{Background}

\section{Computational Effects}

Since the rapid development of computational theory in 1930s by A. Turing, K. Godel and
A. Church, we have a well-established notion of what can and what cannot be done
through algorithmic means, which we can almost directly translate to being computable
by our machines. Through next years we have developed mainstream languages that are used
almost everywhere, with a great success.

Under these circumstances one may pose a question, why do we still bother with
development of languages theory, since so much has been done already. Is there
anything that drives us towards further research? Indeed, one active branch
revolves around equational theory to asses equality of two programs, which we
know that in general setting is undecidable. Proof methods may include extensional,
contextual or logical equality. However, there is no doubt that these formal
ways of reasoning about programs, while being crucial for assessing correctness,
do not bring direct benefits for practical, everyday use cases.

% Not sure I like this part. Isn't it too broad to say handle complexity?
Other branch of languages theory, that we shall investigate more in this thesis,
is about handling complexity of programs. Various methods of static analysis has
been developed, most notably, type systems.

What is the source of complexity? Thanks to strong and static type systems along
with their implementations, we have solid tools to work efficiently on functions
that are pure. That being said, we claim that the sole complexity of programs
comes from side effects, which we cannot avoid in writing anything useful.

We need to have a good way for handling computational effects, which so far has
been modeled through monads~\cite{moggi}.

However, they were found to be, to say the least, cumbersome to work with when
complexity and number of different effects increases. It is perhaps not a coincidence
that many functional programming languages do not have them, because of the
non-composable nature of them.

Having functions $ f: a \rightarrow m b $ and $ g : b \rightarrow m' c $, it
should not be an issue to compose them together.

Recall that the main idea of monads lies on the fact that we provide a function
to lift value into a functor and associative kleisli composition operator to combine
computations together. It turns out, that it becomes complex and unpleasent to
combine two functors together to form a new monad. For this purpose, monad
transformers arose in Haskell.

Most of the common languages avoid this by not expressing computational effects
in the type system, and instead one may think about functions as being implicitly
embed in a Kleisli Category over a functor T, where T is a hidden signature over
all possible side effects that occur in our program.

Not only we would like to bring back effects to our type system, but also do it
in a way that is composable. This is where algebraic effects comes to the rescue.


From theoretical point of view, we need to develop equa about our effects
to assert correctness of our langauge as well as to have the right hammer
to reason about our programs.

That is the point where we would like to introduce to unfamiliar read a notion
of algebraic effects.

\section{Algebraic Effects}

What are algebraic effects?

% Algebraic effects as Interfaces for side effects

Algebraic effects can be thought as an public interface for computational effects.
Declarative approach allow us to write programs in which the actual semantic of
source code is dependent on handler that defines the meaning of a subset of effects.

This is really a breakthrough from practical point of view, as we may substitute
logic depending on execution environment. As an example, fetching for resources
can behave differently as we run tests, debug our code, or run it on production.
In the same manner, they neatly allow us to abstract over implementation details.


\section{Categorical Setting of Universal Algebra}

Algebraic effects can be describe via operational means, however,
for the purpose of presenting the duality between algebra and coalgebra,
we allow ourselves to wander a bit deeper into category theory and describe
effects from denotational point of view.

    \subsection{Algebraic Theories}

    \begin{definition}

    A \textit{signature $ \Sigma $} is given by a collection of operation symbols $ op_{i} $
    with associated parameters $ P_{i} $ and arities $ A_{i} $, where
    $ P_{i} $ and $ A_{i} $ are objects in the category of our interest.
    We will write an operation as $ op_{i} : P_{i} \rightsquigarrow A_{i} $

    \end{definition}

    \begin{definition}

    Collection of $\mathVar{\Sigma-terms}$ is a free algebra with a generator $X$
    for a functor $ \mu H_{\Sigma} $ that maps objects into trees over a given
    signature $ \Sigma $ and morphisms into folds over trees.

    \end{definition}

    \begin{definition}

        A $ \mathVar{\Sigma-Equation} $ is an object X and a pair of
        $\mathVar{\Sigma-terms}$ $l, r \in Tree_{\Sigma}(X)$, written as

        \begin{center}
        $ X \mid l = r $
        \end{center}

    \end{definition}

    \begin{definition}

    An \textit{algebraic theory} $T = (\Sigma_{T}, \mathcal{E}_{T})$, is given by
    a signature $\Sigma_{T}$ and a collection $\mathcal{E}_{T}$ of
    $\mathVar{\Sigma_{T}-equations}$. For clarity we will usually omit T subscript

    \end{definition}

    \begin{definition}

    An \textit{interpretation $I$ over a given signature $\Sigma$} is given by
    a carrier object $|I|$ and for each $ op_{i} : P_{i} \rightsquigarrow A_{i} $
    in $\Sigma$ a map

    \begin{center}
        $ {\llbracket op_{i} \rrbracket}_I : P_{i} \times{} {|I|}^{A_{i}} \rightarrow |I| $
    \end{center}
    Interpretation may be naturally extended to $\mathVar{\Sigma-terms}$, such
    that a given $\mathVar{\Sigma-term}$ $X \mid t$ is interpreted by a map
    which sends variables into projections from environment and terms into
    map composition over each subterm.

    \end{definition}

    \begin{definition}

    A \textit{model M} of an algebraic theory T is an interpretation of the
    signature ${\Sigma_{T}}$ which validates all the equations $\mathcal{E_{T}}$.
    That is, for every equation $X \mid l = r$ the following diagram commutes:

    \begin{center}
    \begin{tikzcd}[row sep=huge, column sep=large]
        {|M|}^k \arrow[rr, '\phi', bend right] \arrow[rr, '\rrbracket{} X|l', bend left] &  & {|M|}
    \end{tikzcd}
    \end{center}
    % TODO: this diagram doesn't work

    \begin{center}
    \begin{tikzcd}
    A \arrow[rd] \arrow[r, '\phi'] & B \\
    & C
    \end{tikzcd}
    \end{center}

    \end{definition}


    \begin{definition}

    Free F-algebra on an object A (of generators) in $\mathcal{C}$ is meant an algebra

    \begin{center}

    $ \varphi_{A} : F A^{\#} \longrightarrow A $

    \end{center}
    \\
    together with an universal arrow $ \eta_{A} : A \longrightarrow A^{\#} $. Universality means that for every algebra
    $ \beta : F B \longrightarrow B $ and every morphism $ f : A \longrightarrow B $ in $ \mathcal{C}$, there exists a unique homomorphism
    $ \overline{f} : A^{\#} \longrightarrow B $ extending f, i.e.\ a unique morphism of $ \mathcal{C}$ for which the diagram below
    commutes:

    % TODO:
    \begin{center}
    \begin{tikzcd}[row sep=huge, column sep=large]
    % F A^{\#} \arrow[d, "F \overline{f}"] \arrow[r, "\varphi_{A}"] & A^{\#} \arrow[d, "\overline{f}"] & A \arrow[l, "\eta_{A}"'] \arrow[ld, "f"] \\
    % F B \arrow[r, "\beta"']                                       & B                                &
    \end{tikzcd}
    \end{center}

    \end{definition}

    \begin{definition} \textit{Free model} is just a model which is free algebra.

    \end{definition}


    \begin{lemma}{Free models form monads}

    \end{lemma}

    \begin{definition}

    Let L, M be models of a theory T. A $ \mathVar{T-homomorphism}$
    $\phi : L \rightarrow M$ is a map such that the following diagram commutes:

    % TODO

    \end{definition}

\section{Duality}
    \subsection{Comodels}

    Comodel in $\mathcal{C}$ is just a Model in $ \mathcal{C}^{op} $.

    Models --- Worlds

    \subsection{Cooperations}

    Derive from Models duality

    \subsection{Coalgebraic Effects}

    \subsection{Coinductive Reasoning}

    Coinduction, and coinductive structures, provide us a way of observation
    of the behaviour, as a contractry to construction in inductive reasoning.
    Simple example of the duality can be expressed through induction over finite
    list and coinductive observation of infinite streams.

    Doing the latter may involve modification of the internal state of the machine
    that is generating the infinite streams, or in more concrete scenario,
    alternation of the external resource that is providing us the data.

    This is one of the cases where interaction with external resource multiple
    times, or more specifically in case of algebraic effects, invoking resumption more
    than once, may lead to unexpected behaviour that would not be expected in standard
    control flow. This captures the excessive generality of effects, and is the
    issue that we would like to address with coalgebraic effects.

    % TODO: On it's own this snippet a bit dry and of course requires additional examples,
    % for instance the classic one with operating on closed file


\section{Related Work}
    \subsection{Algebraic Effects}

    Except from Links language~\cite{handlers-cps}, on which the implementation
    is based, there are currently many other alternatives available. One may take
    a look at Frank~\cite{frank}, which provides a support for multihandlers,
    Koka~\cite{leijen-koka}, Helium~\cite{helium} or Eff~\cite{eff}. Except from
    separate languages, many libraries arose for existing ones like Haskell,
    Idris, Scala or Multicore OCaml.

    As can be seen in the J. Yallop repository~\cite{effects-bibliography}, algebraic
    effects and handlers are now trending branch in the programming languages theory.

    \subsection{Coalgebraic Effects}

    Ahman and Bauer~\cite{runners-in-action}


\chapter{Potential Solutions}

One-shot continuations

Detecting whether resumption is called only once is undecidable.

\section{Dynamic Constraints Checking}

\section{Linear Types}

Solve the issue through introduction of linear type system.

\section{Data-Flow Analysis}

Another static analysis of the program.

\section{Cohandlers as Separate Constructs}

Simplify semantics by separating coalgebraic effects into a new, restricted
construct in programming language

\chapter{Effectful and Coeffectful Programming}
\section{Examples}\label{sec:examples}

    In this section we present a few examples to show the capabilities of the language.
    The ideas have been based on~\cite{programming-in-eff}, and thus will not be
    described in great details. More exemplary programs in Freak language can
    be found under \\ \href{https://github.com/Tomatosoup97/freak/tree/master/src/programs}{\underline{https://github.com/Tomatosoup97/freak/tree/master/src/programs}}.

    \subsection{Choice}\label{sec:choice-example}

    The first example will be based on modelling (nondeterministic) choice
    in the program. We will make two decisions, which will affect the computation
    result:

    \begin{verbatim}
    let c1 <- do Choice () in
    let c2 <- do Choice () in
    let x <- if c1 then return 10 else return 20 in
    let y <- if c2 then return 0 else return 5 in
        return x - y
    \end{verbatim}
    With that in hand, we may want to define effect handlers:

    \begin{verbatim}
    handle ... with {
        Choice p r ->
            let t <- r 1 in
            let f <- r 0 in
            <PLACEHOLDER> |
        return x -> return x
    }
    \end{verbatim}
    where in the \verb!<PLACEHOLDER>! we can define on what to do with the
    computation. For example, min-max strategy for picking the minimum value:

    \begin{verbatim}
    if t < f then return t else return f
    \end{verbatim}
    where the code evaluates to \verb!5!. Another example is a handler that
    collects all possible results, which can be achieved by putting
    \verb!return (t, f)! in the \verb!<PLACEHOLDER>!, which evaluates to
    \verb!((10, 5), (20, 15))!.

    \subsection{Exceptions}

    Exceptions are simply algebraic effect handlers which drop the resumption.

    \begin{verbatim}
    handle
        if x == 0 then do ZeroDivisionError ()
                  else return 1/x
    with {
        ZeroDivisionError p r -> return 42 |
        return x -> return x
    }
    \end{verbatim}
    Where we imagine that $x$ variable has been bound previously.

    \subsection{Taming Side effects}

    The complexity of the programs and their performance usually comes from side effects.
    Algebraic effects allow us to define code in a declarative manner, and hence
    neatly tame the side effects that they produce. This gives us a lot of flexibility
    in the actual meaning without duplicating the code. Let's consider the following
    very basic code snippet:

    \begin{verbatim}
    let x <- do Fetch () in
    -- operate on x
    \end{verbatim}

    The code is dependent on a context in which it is executed, which here is
    the handler that defines the behaviour of the algebraic \verb!Fetch! effect.
    In the imperative, or even functional approach, we would need to provide
    the interface for fetching the data by doing dependency injection or even
    embedding the operation directly. Here we are just stating what operation
    we are performing, leaving the interpretation up to the execution context,
    which could do the fetching or mock the external resource.

    These implications are straightforward when looking from a categorical standpoint,
    where effects are viewed as free models of algebraic theories~\cite{adequacy},
    and handlers are homomorphisms preserving the model structure~\cite{handlers}.
    Nevertheless, the results are very exciting for programming use cases. The current
    Freak implementation does not support I/O.

\section{Coexamples}

    Examples for cohandlers

\section{Usage guide}

    % Outdated
    As of this day, two implementations are available, one based on the curried
    translation and Appel~\cite{appel-continuations}, and the second one based
    directly on the uncurried translation with continuations as explicit stacks
    from paper. More details can be found in Section~\ref{sec:implementation}.
    All commands are available within \verb!src!  directory.

    \subsection{Build and install}

    \begin{itemize}
        \item Install dependencies: \verb!make install!
        \item Select implementation: \verb!make link-lists! (default) vs \verb!make link-appel!
        \item Compile: \verb!make build!
        \item Link to PATH:~\verb!sudo make link!
        \item Remove artifacts: \verb!make clean!
    \end{itemize}

    After compiling and linking program to PATH, one may evaluate program as
    follows: \verb!freak programs/choicesList.fk!. The actual code is described in Section~\ref{sec:choice-example}

    \subsection{Running tests}

    Test cases are available \href{https://github.com/Tomatosoup97/freak/blob/master/src/Tests.hs}{\underline{here}},
    they include both inline and file-based tests. For more details about
    writing tests, one may refer to \textit{HUnit documentation}~\cite{hunit-docs}.

    \begin{itemize}
        \item Run tests: \verb!make tests!
        \item Run code linter: \verb!make lint!
        \item Compile, run linter and tests: \verb!make check!
    \end{itemize}

\chapter{Calculus of Freak language}
\section{Syntax}

    The syntax for the calculus is shown below. $nat \; n$ represents an integer $n$,
    $V \oplus W$ and $V \approx W$ are respectively binary and relational operators,
    where we support basic arithmetic and comparison operations.
    \textbf{if} $V$ \textbf{then} $M$ \textbf{else} $N$ is a standard branching
    statement. The other constructs are just as in Links, with slight syntax
    modifications. Actual programs in Freak can be found in Section~\ref{sec:examples}.

    \begin{grammar}

        <Values V, W> $::=$ $ x $ | $nat \; n$ \\
            | $ \backslash x : A \rightarrow M $ | \textbf{rec} $ g \; x \rightarrow M $\\
            | $V \oplus W$ | $V \approx W$ \\
            | <> | $ \{ \ell = V; W\} $  | ${[ \ell \; V]}^{R}$

        <Computations M, N> $::=$ $ V $ $ W $ \\
            | \textbf{if} $V$ \textbf{then} $M$ \textbf{else} $N$ \\
            | \textbf{let} $\{\ell  = x; y\} = V$ \textbf{in} $ N $ \\
            | \textbf{case} $V \{ \ell \; x \rightarrow M; y \rightarrow N\}$ | \textbf{absurd} $ V $ \\
            | \textbf{return} $V$ | \textbf{let} $ x \leftarrow M $ \textbf{in} $ N $ \\
            | \textbf{do} $\ell \; V$ | \textbf{handle} $M$ \textbf{with} $ \{ H \} $

        <Handlers H> $::=$ \textbf{return} $ x \rightarrow M $ | $ \ell \; p \; r \rightarrow M, H $

        <Binary operators $\oplus$> $::=$ + | $-$ | * | /

        <Relational operators $\approx$> $::=$ $ \textless $ | $\leqslant$ | $>$ | $\geqslant$ | == | $!= $

    \end{grammar}

\section{Typing Rules}
\section{Operational Semantics}

    The source language's dynamics have been described
    extensively by providing small-step operational semantics,
    continuation passing style transformation~\cite{handlers-cps} as well
    as abstract machine~\cite{liberating-effects}, which was proved to coincide
    with CPS translation. That being said, Freak introduces new basic
    constructs to the language, for which we shall define the semantics.

    \begin{flushleft}
    Extension of the evaluation contexts:
    \end{flushleft}

    \begin{flushleft}
    $\mathcal{E} ::= \mathcal{E} \oplus W \; | \; nat \; n \oplus \mathcal{E} \; |$ \textbf{if} $\mathcal{E}$ \textbf{then} $M$ \textbf{else} $N$
    \end{flushleft}

    \begin{flushleft}
    Small-step operational semantics:
    \end{flushleft}

    \begin{flushleft}
    \textbf{if} $nat \; n$ \textbf{then} $M$ \textbf{else} $N \rightsquigarrow M \quad \quad $ if $n \neq 0$ \\
    \textbf{if} $nat \; n$ \textbf{then} $M$ \textbf{else} $N \rightsquigarrow N \quad \quad $ if $n = 0$
    \end{flushleft}

    \begin{flushleft}
    $nat \; n \oplus nat \; n' \rightsquigarrow n'' \quad \quad $    if $ n'' = n \oplus n' $ \\
    $nat \; n \approx nat \; n' \rightsquigarrow 1  \quad \quad $    \; if $ n \approx n' $ \\
    $nat \; n \approx nat \; n' \rightsquigarrow 0  \quad \quad $    \; if $ n \not\approx n' $

    \end{flushleft}

\section{Continuation Passing Style Transformation}
    \ldots

\chapter{Implementation}\label{sec:implementation}

    The Freak implementation is available at \href{https://github.com/Tomatosoup97/freak}{\underline{https://github.com/Tomatosoup97/freak}},
    written purely in Haskell. While the paper provided a good overview of the
    language and the translation, the lower-level details were omitted. That
    being said, two inherently different takes at the implementations were made.
    The first one is based on curried translation and A. Appel~\cite{appel-continuations}
    book, and the second one directly on the uncurried translation to target
    calculus with continuations represented as explicit stacks from the paper.
    We start by presenting core data structures, and afterwards move to actual
    translation details.

    \section{Abstract Syntax Tree}\label{sec:implementation-ast}

    The language's AST is defined without surprises, just as syntax is:

    \begin{verbatim}
data Value
    = VVar Var
    | VNum Integer
    | VLambda Var ValueType Comp
    | VFix Var Var Comp
    | VUnit
    | VPair Value Value
    | VRecordRow (RecordRow Value)
    | VExtendRow Label Value Value
    | VVariantRow (VariantRow Value)
    | VBinOp BinaryOp Value Value

data Comp
    = EVal Value
    | ELet Var Comp Comp
    | EApp Value Value
    | ESplit Label Var Var Value Comp
    | ECase Value Label Var Comp Var Comp
    | EReturn Value
    | EAbsurd Value
    | EIf Value Comp Comp
    | EDo Label Value
    | EHandle Comp Handler
    \end{verbatim}
    Similarly for the target calculus data structure. However, as one may notice,
    for convenience the \textbf{let} translation is homomorphic, as opposed to be
    to lambda abstracted with immediate application:

    \begin{verbatim}
data UValue
    = UVar Var
    | UNum Integer
    | UBool Bool
    | ULambda Var UComp
    | UUnit
    | UPair UValue UValue
    | ULabel Label
    | URec Var Var UComp
    | UBinOp BinaryOp UValue UValue

data UComp
    = UVal UValue
    | UApp UComp UComp
    | USplit Label Var Var UValue UComp
    | UCase UValue Label UComp Var UComp
    | UIf UValue UComp UComp
    | ULet Var UComp UComp
    | UAbsurd UValue

    \end{verbatim}
    The final answer, common to both evaluations, is represented as a \verb!DValue!,
    where the meaning of the coproduct is as one would expect:

    \begin{verbatim}
type Label = String
type FuncRecord = [DValue] -> Either Error DValue
data DValue
    = DNum Integer
    | DLambda FuncRecord
    | DUnit
    | DPair DValue DValue
    | DLabel Label
    \end{verbatim}

    \section{Target calculus}

    \section{Curried translation}

    The first take was heavily inspired by A. Appel's Compiling with
    Continuations~\cite{appel-continuations}, which provides a translation for
    a simplified ML calculus. The calculus was extended and translation adapted to
    handle algebraic effects and their handlers. The translation is based on the curried
    first-order translation. That being said, the source code diverged a lot from the
    paper on which it was based, leading to a different transformation for which the
    correctness and cohesion with operational semantics should be proved separately.
    Indeed, while the interpreter worked well on the use cases defined in tests, the
    evaluation had a part which was not tail-recursive. What's more, nested handlers
    were not supported, and the implementation was found to be trickier than it should,
    as it was not obvious on how to adopt the technique proposed in the paper.

    In terms of improving the performance of the evaluation, uncurried higher-order
    translation should be adapted, so that administrative redexes are contracted
    and proper tail-recursion is obtained. The core data structure, into which the
    source program is transformed, is defined as follows:

    \begin{verbatim}
data ContComp
    = CPSApp CValue [CValue]
    | CPSResume CValue ContComp
    | CPSFix Var [Var] ContComp ContComp
    | CPSBinOp BinaryOp CValue CValue Var ContComp
    | CPSValue CValue
    | CPSLet Var CValue ContComp
    | CPSSplit Label Var Var CValue ContComp
    | CPSCase CValue Label Var ContComp Var ContComp
    | CPSIf CValue ContComp ContComp
    | CPSAbsurd CValue
    \end{verbatim}
    Most of the terms at the end have a coinductive reference to itself, which
    represents the rest of the computation that needs to be done. For more
    clarification, one may take a look into the book mentioned
    above~\cite{appel-continuations}. The source code for curried translation and
    evaluation can be found respectively in \verb!CPSAppel.hs! and
    \verb!EvalCPS.hs!.

    \section{Uncurried translation}

    Having in mind the drawbacks mentioned above, alternative translation was
    written, that coincides with the translation from the paper. Namely, with
    the uncurried translation to target calculus with continuations represented
    as explicit stacks. The target calculus was described in Section~\ref{sec:implementation-ast},
    for which the evaluation is straightforward. The continuations are represented
    as \verb!Cont!, with syntactic distinction between pure and effectful computations,
    which occupy alternating positions in the stack. Explicit distinction gave
    more control in the source code.

    \begin{verbatim}
type CPSMonad a = ExceptT Error (State Int) a

type ContF = UValue -> [Cont] -> CPSMonad UComp

data Cont = Pure ContF
          | Eff ContF
    \end{verbatim}
    Where \verb!CPSMonad! is a monad transformer over \verb!Either! and \verb!State!.
    \verb!State! was required to generate labels for fresh variables that
    came from the translation. The core code is split into five functions:

    \begin{verbatim}
cps     :: Comp    -> [Cont] -> CPSMonad UComp
cpsVal  :: Value   -> [Cont] -> CPSMonad UValue
cpsHRet :: Handler -> Cont
cpsHOps :: Handler -> Cont
forward :: Label   -> UValue -> UValue -> [Cont] -> CPSMonad UComp
    \end{verbatim}
    Where the first two are implementing cps for computations and values.
    \verb!cpsHRet! and \verb!cpsHOps! are yielding pure and effectful continuations,
    based on a given handler. The last one is responsible for forwarding the
    computation to the outer handler.

    This results in an implementation that finally supports nested handlers,
    as can be seen by evaluating \verb!programs/complexNestedHandlers.fk! program.
    Unfortunately, following closely translation from the paper resulted in
    a behaviour, in which invoked resumption forgets its pure continuation. This means,
    that the following code, evaluating correctly on the Appel-based translation,
    returns 0 rather than 1:

    \begin{verbatim}
handle do Drop () with { Drop p r -> let t <- r 0 in return 1}
    \end{verbatim}
    Nevertheless, working out this issue appears as less demanding than coping with
    discrepancies created in the first translation. The source code for uncurried
    translation and evaluation can be found respectively in \verb!CPSLists.hs! and
    \verb!EvalTarget.hs!.

    \section{Cohandlers}
    \ldots

    \section{Source Code Structure}

    The source code is divided into a number of modules, where the most
    crucial parts have already been described.

    \begin{verbatim}
    AST.hs          - AST data structures
    CommonCPS.hs    - Common functions for CPS translation
    CommonEval.hs   - Common functions for evaluation
    CPSAppel.hs     - Appel-based CPS translation
    CPSLists.hs     - Uncurried CPS translation
    EvalCPS.hs      - Evaluation of the Appel's CPS structure
    EvalTarget.hs   - Evaluation of the target calculus
    Freak.hs        - API for the language
    Main.hs         - Main module running evaluator on given filename
    Parser.hs       - Parser and lexer
    TargetAST.hs    - AST for the target calculus
    Tests.hs        - Tests module
    Types.hs        - Common types definition
    programs/       - Exemplary programs used in tests
    \end{verbatim}

\chapter{Conclusion}
\section{Summary}

    What was achieved, what was described, what needs to be done or researched.

\section{Future work}

    \subsection{Abstract machine}

    The Links language also provides small-step operational semantics and
    an abstract machine~\cite{liberating-effects}. Implementing another way
    of evaluation could serve as a way to empirically assert correctness,
    as opposed to formally.

    \subsection{Type inference and row polymorphism}

    The type system as of this day is not implemented, as the focus has been put
    on CPS transformation. Further work is required here, especially considering
    the fact that a huge advantage of algebraic effects is that they are explicitly
    defined in the type of a computation.

    \subsection{Multiple instances of algebraic effects}

    The Freak language is limited to a single instance of an effect. We would
    need to support cases where many instances of the algebraic effects, with
    the same handler code, could be instantiated. The current state of the
    art introduces a concept of resources and instances, as in Eff~\cite{programming-in-eff},
    or instance variables, as in Helium~\cite{binders-labels}.

    \subsection{Selective CPS}

    Other languages, like Koka~\cite{leijen-koka}, or even the core of the Links, are
    performing selective CPS translation, which reduces the overhead on code
    that does not perform algebraic effects. Our current translation is fully
    embedded in the CPS.\@

    \subsection{Exceptions as separate constructs}

    Exceptions are a trivial example of algebraic effect where the resumption is
    discarded, and as described in \S 4.5~\cite{handlers-cps}, they can be modeled
    as a separate construct to improve performance.

    \subsection{Shallow handlers}

    Shallow and deep handlers while being able to simulate each other up to
    administrative reductions, have a very different meaning from a theoretical point
    of view. Implementing them as defined by Lindley et al.~\cite{shallow-handlers}
    could be another way of enhancing Freak.


%%%%% BIBLIOGRAPHY

\printbibliography{}

\end{document}
